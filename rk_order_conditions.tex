\documentclass[12pt]{article}

\usepackage[T1]{fontenc}
\linespread{1.05}         % Palatino needs more leading (space between lines)
\usepackage[hmargin=1.0in]{geometry}

\usepackage{graphicx}
\usepackage{amsmath,amssymb,amsthm}

\bibliographystyle{apalike}

\begin{document}

\title{Order conditions for Runge-Kutta methods}
\author{David I. Ketcheson} 
\maketitle


\newtheorem{thm}{Theorem}
\newtheorem{dfn}{Definition}
\newtheorem{lem}{Lemma}
\newtheorem{cor}{Corollary}
%\newtheorem{proposition}{Proposition}
%\newtheorem{result}{Result}
%\newtheorem{proof}{Proof}
%\newtheorem{conj}{Conjecture}
\newcommand{\qh}{\hat{q}}
\newcommand{\be}{\begin{equation}}
\newcommand{\ee}{\end{equation}}
\newcommand{\bq}{\mathbf{q}}
\newcommand{\bx}{\mathbf{x}}
\newcommand{\by}{\mathbf{y}}
\newcommand{\br}{\mathbf{r}}
\newcommand{\imh}{{i-\frac{1}{2}}}
\newcommand{\iph}{{i+\frac{1}{2}}}
\newcommand{\ipmh}{{i \pm \frac{1}{2}}}
\newcommand{\jph}{{j+\frac{1}{2}}}
\newcommand{\Aop}{{\cal A}}
\newcommand{\Bop}{{\cal B}}
\newcommand{\Wop}{{\cal W}}
\newcommand{\Oop}{{\cal O}}
\newcommand{\DQ}{\Delta Q}
\newcommand{\Dq}{\Delta q}
\newcommand{\Dx}{\Delta x}
\newcommand{\Dy}{\Delta y}
\newcommand{\Du}{\Delta u}
\newcommand{\bu}{\mathbf{u}}
\newcommand{\bv}{\mathbf{v}}
\newcommand{\bw}{\mathbf{w}}
\newcommand{\bU}{\mathbf{U}}
\newcommand{\bV}{\mathbf{V}}
\newcommand{\bF}{\mathbf{F}}
\newcommand{\Lop}{{\cal L}}
\newcommand{\Sop}{{\cal S}}
\newcommand{\Fop}{{\cal F}}
\newcommand{\Dofr}{{\cal D}(r)}
\newcommand{\Dt}{\Delta t}
\newcommand{\bbA}{\mathbf{A}}
\newcommand{\bbZ}{\mathbf{Z}}
\newcommand{\bbK}{\mathbf{K}}
\newcommand{\bbI}{\mathbf{I}}
\newcommand{\bbb}{\bar{\mathbf{b}}}
\newcommand{\bbe}{\mathbf{e}}
\newcommand{\bbone}{\mathbf{1}}
\newcommand{\lnorm}{\left\|}
\newcommand{\rnorm}{\right\|}

\newcommand{\dx}{\Delta x}
\newcommand{\dt}{\Delta t}
\newcommand{\aij}{\alpha_{i,j}}
\newcommand{\bij}{\beta_{i,j}}
\newcommand{\lt}{\tilde{L}}

\newcommand{\hf}{\frac{1}{2}}
\def\half{\frac{1}{2}}
\newcommand{\fracStrut}{\rule[-1.0ex]{0pt}{3.1ex}}
\newcommand{\hfs}{\ensuremath{\frac{1}{2}}\fracStrut}
\newcommand{\scinot}[2]{\ensuremath{#1\times10^{#2}}}
\newcommand{\dee}{\mathrm{d}}
\newcommand{\dye}{\partial}
\newcommand{\diff}[2]{\frac{\dee #1}{\dee #2}}
\newcommand{\pdiff}[2]{\frac{\dye #1}{\dye #2}}
\newcommand{\Real}{\mathbb{R}}
\newcommand{\Complex}{\mathbb{C}}
% matrices
\newcommand{\m}[1]{\mathbf{#1}}
\newcommand{\mA}{\m{A}}
\newcommand{\mAb}{\bar{\m{A}}}
\newcommand{\vah}{\hat{\v{a}}}
\newcommand{\mAh}{\hat{\m{A}}}
\newcommand{\mD}{\m{D}}
\newcommand{\mS}{\m{S}}
\newcommand{\mT}{\m{T}}
\newcommand{\mR}{\m{R}}
\newcommand{\mP}{\m{P}}
\newcommand{\mM}{\m{M}}
\newcommand{\mQ}{\m{Q}}
\newcommand{\mI}{\m{I}}
\newcommand{\mK}{\m{K}}
\newcommand{\mL}{\m{L}}
\newcommand{\mzero}{\m{0}}
% these use the upgreek package to get non-italic greek, which doesn't
% seem to work with \mathbf so these have to be setup manually to
% match \m
\newcommand{\matalpha}{\boldsymbol{\upalpha}}
\newcommand{\matbeta}{\boldsymbol{\upbeta}}
\newcommand{\mattheta}{\boldsymbol{\uptheta}}
\newcommand{\mateta}{\boldsymbol{\upeta}}
\newcommand{\matmu}{\boldsymbol{\upmu}}
\newcommand{\matlambda}{\boldsymbol{\uplambda}}
\newcommand{\matgamma}{\boldsymbol{\upgamma}}
\newcommand{\matdelta}{\boldsymbol{\updelta}}
% vectors
\renewcommand{\v}[1]{\boldsymbol{#1}}
\newcommand{\transpose}{^\mathrm{T}}
\newcommand{\thT}{\mattheta\transpose}
\newcommand{\bT}{\v{b}\transpose}
\newcommand{\bh}{\hat{b}}
\newcommand{\vbh}{\hat{\v{b}}}
\newcommand{\bhT}{\hat{\v{b}}\transpose}
\newcommand{\vb}{\v{b}}
\newcommand{\vc}{\v{c}}
\newcommand{\ve}{\v{e}}
\newcommand{\vu}{\v{u}}
\newcommand{\vl}{\boldsymbol{l}}
\newcommand{\vv}{\v{v}}
\newcommand{\vy}{\v{y}}
\newcommand{\vd}{\v{d}}
\newcommand{\va}{\v{a}}
\newcommand{\vf}{\v{f}}
\newcommand{\Matlab}{{\sc Matlab}\xspace}
\newcommand{\BARON}{{\sc Baron}\xspace}
\newcommand{\code}[1]{\textsf{#1}}
% the SSP coefficient
\newcommand{\sspcoef}{\mathcal{C}}
\newcommand{\clin}{\sspcoef_{\textup{lin}}}
\newcommand{\ceff}{\sspcoef_{\textup{eff}}}
\newcommand{\DtFE}{\Dt_{\textup{FE}}}

\newcommand{\Inc}{\textrm{Inc}}

\newcommand{\ty}{\tilde{y}}
\newcommand{\tu}{\tilde{u}}
\newcommand{\pt}{\tilde{p}}
\newcommand{\lte}{\tau}
\newcommand{\ste}{\boldsymbol{\tau}}
\newcommand{\bgamma}{\boldsymbol{\gamma}}
\newcommand{\btheta}{\boldsymbol{\theta}}
\newcommand{\bty}{\mathbf{\ty}}
\newcommand{\btu}{\mathbf{\tilde{u}}}

\newcommand{\gerr}{\delta u}
\newcommand{\gserr}{\delta\boldsymbol{y}}
\newcommand{\gsteperr}{\bar{\boldsymbol{\epsilon}}}
\newcommand{\rhserr}{\delta}
\newcommand{\rhsserr}{\delta\boldsymbol{f}}
\newcommand{\deltaset}{\bar{\boldsymbol{\delta}}}

\newcommand{\bff}{\mathbf{f}}
\newcommand{\bFf}{\mathbf{F}}
\newcommand{\bfb}{\mathbf{f}_u}
\newcommand{\bfbt}{\tilde{\mathbf{f}}_u}
\newcommand{\btf}{\mathbf{\tilde{f}}}
\newcommand{\bone}{\mathbf{1}}
\newcommand{\bb}{\mathbf{b}}
\newcommand{\bbh}{\hat{\mathbf{b}}}
\newcommand{\bc}{\mathbf{c}}
\newcommand{\bt}{\mathbf{t}}
\newcommand{\bg}{\mathbf{g}}
\newcommand{\bfe}{\mathbf{e}}
\newcommand{\mC}{\m{C}}

\renewcommand{\v}[1]{\mathbf{#1}}


A Runge--Kutta method 
$$
u^{n+1} = u^n + \Delta t \sum_j b_j f(y_j, t^n + c_j \Dt)
$$
is similar to a quadrature rule with weights $b_j$ and
abscissas $c_j$.  But the Runge--Kutta method is more complex, since 
intermediate approximations must be performed to obtain the stage values
$y_j$, meaning that the values $f(y_j)$ appearing in the quadrature formula
are not exact derivatives of the
solution.  The order conditions for Runge--Kutta methods are precisely 
the necessary conditions to ensure that the resulting errors in the final
approximation vanish up to a given order in $\Delta t$.

\section{Derivation of the order conditions\label{sec:rkoc}}
  In this section we review the derivation of order conditions for Runge-Kutta
methods via the approach due to Albrecht \cite{albrecht1996}.
The basic notation is defined in section \ref{notation}.  A formula 
for the global error in terms of the method coefficients is derived
in section \ref{globerr}.  This leads to a recursion for determining the order 
conditions, as explained in section \ref{recursion}.

\subsection{Notation\label{notation}}
Write the $n$th step of a Runge--Kutta method as
\begin{subequations} \label{eq:rk}
\begin{align} 
\by^n & = u^{n}\ve + \Dt \mA \bff^{n} \\
u^{n+1} & = u^{n} + \Dt \bb\transpose \bff^{n}
\end{align}
\end{subequations}
where 
\begin{align*}
\by^n & = [y_1^n,\dots,y_s^n], \\
\bff^n & = [F(y_1^n),\dots,F(y_s^n)]
\end{align*}
are the vector of stage values and stage derivatives, respectively.
Let $u(t)$ denote the exact solution at time $t$ and define
the stage time vector $t^{n+c}_i = t^n+c_i\Dt$ and
the vectors of exact stage solution values and
exact stage derivatives:
\begin{align*}
    \by(\bt^{n+c}) & = [u(t^n+c_1\Dt),\dots,u(t^n+c_s\Dt)], \\
    \btf(\bt^{n+c}) & = [F(u(t^n+c_1\Dt)),\dots,F(u(t^n+c_s\Dt))].
\end{align*}
Next define the {\em truncation error} $\lte^n$ and
{\em stage truncation errors} $\ste^n$ by
\begin{subequations} \label{eq:rk_lte}
\begin{align} 
    \by(\bt^{n+c}) & = u(t^{n})\ve + \Dt \mA \btf(\bt^{n+c}) + \Dt \ste^n \\
    u(t_{n+1}) & = u(t^{n}) + \Dt \bb\transpose\btf(\bt^{n+c}) + \Dt \lte^n.
\end{align}
\end{subequations}

\subsection{Relation between the global and local errors\label{globerr}}
To find formulas for the truncation errors, we make use of 
the Taylor expansions
\begin{subequations} \label{eq:taylor}
\begin{align} 
u(t^{n}+c_i\Dt) & = \sum_{k=0}^\infty \frac{1}{k!} \Dt^k c_i^k u^{(k)}(t^{n}) \\
F(u(t^{n}+c_i\Dt)) & = \sum_{k=1}^\infty \frac{1}{(k-1)!} \Dt^{k-1} c_i^{k-1} u^{(k)}(t^{n})
\end{align}
\end{subequations}

Substitution of \eqref{eq:taylor} into \eqref{eq:rk_lte} gives
\begin{subequations} \label{eq:ste}
\begin{align}
\ste^n & = \sum_{k=1}^\infty \left( \frac{1}{k!} \bc^k - \frac{1}{(k-1)!} \mA \bc^{k-1}\right) \Dt^{k-1} u^{(k)}(t^{n}) = \sum_{k=1}^\infty \ste_k \Dt^{k-1} u^{(k)}(t^{n}) \\
\lte^n & = \sum_{k=1}^\infty \left( \frac{1}{k!} - \frac{1}{(k-1)!} \bb\transpose \bc^{k-1}\right) \Dt^{k-1} u^{(k)}(t^{n}) = \sum_{k=1}^\infty \lte_k \Dt^{k-1} u^{(k)}(t^{n})
\end{align}
\end{subequations} 
where 
\begin{subequations} 
\begin{align} \label{eq:ste_coeffs}
\ste_k & = \frac{1}{k!} \bc^k - \frac{1}{(k-1)!} \mA \bc^{k-1} \\
\lte_k & = \frac{1}{k!} - \frac{1}{(k-1)!} \bb\transpose \bc^{k-1}
\end{align}
\end{subequations} 

Subtracting \eqref{eq:rk_lte} from \eqref{eq:rk} gives
\begin{subequations} \label{eq:gserr_series}
\begin{align}
\gserr^n & = \gerr^{n} \ve + \Dt \mA \rhsserr^n - \Dt \ste^n \\
\gerr^{n+1} & = \gerr^{n} + \Dt \bb\transpose \rhsserr^n - \Dt \lte^n,
\end{align}
\end{subequations}
where $\gerr^{n+1}=u^{n+1}-u(t^{n+1})$ is the global error,
$\gserr^n = \by^n-\by(\bt^{n+c})$, is the global stage error, and
$\rhsserr^n = \bff^n-\bff(t^{n+c})$ is the stage derivative error.

Next assume expansions for the stage derivative errors $\rhsserr^n$ and
stage errors $\gserr^n$ as a power series in $\Dt$:
\begin{subequations} \label{eq:err_series}
\begin{align} \label{eq:rhserr_series}
\rhsserr^n & = \sum_{k=0}^{\infty} \rhsserr_k^n \Dt^{k}, \\
\gserr^n & = \sum_{k=0}^{\infty} \gserr_k^n \Dt^{k}.
\label{eq:gserr_power_series}
\end{align}
\end{subequations}
Then substituting the expansions \eqref{eq:rhserr_series} and \eqref{eq:ste}
into the global error formula \eqref{eq:gserr_series} yields
\begin{subequations} \label{eq:gserr_rec_both}
\begin{align} \label{eq:gserr_rec}
\gserr^n & = \gerr^n \ve + \sum_{k=0}^{p-1} \mA \rhsserr^n_k \Dt^{k+1} -\sum_{k=1}^{p} \ste_k u^{(k)}(t^{n}) \Dt^k + \Oop(\Dt^{p+1}) \\
\gerr^{n+1} & = \gerr^n + \sum_{k=0}^{p-1} \bb\transpose\rhsserr^n_{k-1} \Dt^{k+1} -\sum_{k=1}^{p} \lte_k u^{(k)}(t^{n}) \Dt^k + \Oop(\Dt^{p+1}) 
\end{align}
\end{subequations}
Assuming stable propagation of errors, we have global accuracy of order $p$
if the following conditions hold:
\begin{align*}    
\lte_k & = 0  & \mbox{for }  & 1\le k\le p \\
\bb\transpose\rhsserr^n_k & = 0 &  \mbox{for } & 0\le k\le p-1.
\end{align*}
The first set of conditions simply say that our quadrature rule would be of order $p$
if the stage values were exact.
The second set of conditions ensure that the errors due to the stage approximations
cancel out up to order $p$.

It remains to determine the vectors $\rhsserr^n_k$. In fact, 
we can relate these recursively to the global stage error vectors $\gserr_k$.  First define
\begin{align*}
\bFf(\by,\bt) & = [F(y_1(t^n+c_1 \Delta t)),\dots,F(y_s(t^n + c_s \Delta t))]\transpose.
\end{align*}
Then we have the Taylor series
\begin{align*}
    \bff^n =  \bFf(\by^n,\bt^{n+c}) & = \bff(\bt^{n+c}) + \sum_{j=1}^\infty \frac{1}{j!} (\by^n-\by(\bt^{n+c}))^j \cdot
    \bFf^{(j)}(\by(\bt^{n+c}),\bt^{n+c}) \\
    & = \bff(\bt^{n+c}) + \sum_{j=1}^\infty \frac{1}{j!} (\gserr^n)^j \cdot \bg_j(\bt^{n+c}),
\end{align*}
where 
\begin{align*}
\bFf^{(j)}(\by,\bt) = [F^{(j)}(y_1(t^n + c_1\Delta t)),\dots,F^{(j)}(y_s(t^n + c_s \Delta t))]\transpose, \\
\bg_j(\bt) = [F^{(j)}(y(t^n + c_1 \Delta t)),\dots,F^{(j)}(y(t^n + c_s \Delta t))]\transpose,
\end{align*}
and the dot product denotes componentwise multiplication. Thus
\begin{align}
  \rhsserr^n = \bff^n - \bff(\bt^{n+c}) = \sum_{j=1}^\infty \frac{1}{j!} (\gserr^n)^j \cdot
          \bg_j(\bt^{n+c}).
\end{align}
Since
\begin{align}
    \bg_j(\bt^{n+c}) & = \sum_{l=0}^\infty \frac{\Dt^l}{l!} \mC^l \bg_j^{(l)}(t^{n}),
\end{align}
where $\mC=\textup{diag}(\bc)$, we finally obtain the desired expansion:
\begin{align} \label{eq:triplesum}
    \sum_{k=0}^\infty \rhsserr^n_k & = \sum_{k=0}^\infty \sum_{j=1}^\infty \frac{\Dt^k}{j!k!} \mC^k (\gserr^n)^j \cdot \bg_j^{(k)}(t^{n}).
\end{align}

\subsection{Generation of stage derivative error vectors\label{recursion}}
Combining \eqref{eq:gserr_power_series} with \eqref{eq:gserr_rec} and equating
coefficients of powers of $\Dt$ gives (for $k\ge 1$)
\begin{align} \label{eq:gserrk}
\gserr^n_k & = \mA \rhsserr^n_{k-1} - \ste_k \tu^{(k)}(t^n).
\end{align}

To determine the coefficients $\rhsserr_k$, we alternate recursively 
between \eqref{eq:triplesum} and \eqref{eq:gserrk}.
Typically, the abscissas $\vc$ are chosen as $\mA\ve$ so that $\ste_1=0$; 
we will assume this since it simplifies the conditions considerably.

The terms appearing in the $\rhsserr_k$ involve products of certain constants
with derivatives of $\tu$ and the Butcher coefficients.  In order for $\bb\transpose \delta_k$
to vanish for arbitrary $\tu$, it must be that $\bb\transpose \bv = 0$ for each vector $\bv$
appearing in $\rhsserr_k$.  Since this latter condition does not depend on $|\bv|$,
the constants and the derivatives of $\tu$ can be neglected in our analysis.
Hence we focus solely on the vectors appearing in $\rhsserr_k$ depending on the 
Butcher coefficients.  We use the symbol $\deltaset_k$ to denote the set of these vectors.
Then the order conditions for order $p$ can be summarized as follows:
\begin{subequations}
\begin{align}\label{eq:bushy}
\frac{1}{k!} & = \frac{1}{(k-1)!}\bb\transpose \bc^{k-1} 
            & \mbox{for } 1\le k\le p \\
\bb\transpose\bv & = 0 & \mbox{for all } \bv\in\deltaset_k, \ \ \ \ \mbox{for } 1\le k\le p-1.
\label{eq:nonbushy}
\end{align}
\end{subequations}
%where $\bc=\mA\ve$ and
%\begin{align} 
%\ste_k & = \frac{1}{k!} \bc^k - \frac{1}{(k-1)!} \mA \bc^{k-1}.
%\end{align}
The conditions \eqref{eq:bushy} are referred to as {\em bushy-tree order conditions}
because they are associated with the bushy trees in Butcher's approach \cite{butcher2003}.
It is convenient to refer to the order conditions \eqref{eq:nonbushy} as {\em non-bushy-tree
order conditions}; the remainder of the section focuses on the task of determining these
explicitly.
Also $\rhsserr_0=0$ for any consistent method. Then taking $k=1$
in \eqref{eq:gserrk} gives $\gserr_1=0$.  Plugging this into \eqref{eq:triplesum}
yields $\rhsserr_1=0$.

Taking $k=2$ in \eqref{eq:gserrk}, we see that the factor $\ste_2$ appears in $\gserr_2$.
Plugging this into \eqref{eq:triplesum}, we see that $\ste_2$ appears in $\deltaset_2$.
Using this (with $k=3$) in \eqref{eq:gserrk}, we have that $\mA\ste_2$ and $\ste_3$
appear in $\gserr_3$.  Substituting this into \eqref{eq:triplesum} reveals that terms
proportional to $\mC\ste_2, \mA\ste_2, \ste_3$ appear in $\deltaset_3$.

Proceeding in this manner, the order conditions for any order of accuracy can be obtained.





\section{Enumeration of conditions}
  In this section we write out explicitly (for reference) the result of applying
the recursion derived in the previous section.

\subsection{Terms appearing in the error vectors}
Here we enumerate the terms generated by the recursion outlined above.
As before, we assume $\bc=\mA\bfe$ so that $\ste_1=0$.  
\begin{itemize}
  \item Terms appearing in $\deltaset_1$: $\emptyset$
  \item Terms appearing in $\gserr_2$: $\ste_2$
  \item Terms appearing in $\deltaset_2$: $\ste_2$
  \item Terms appearing in $\gserr_3$: $\mA\ste_2,\ste_3$
  \item Terms appearing in $\deltaset_3$: $\mC\ste_2, \mA\ste_2, \ste_3$
  \item Terms appearing in $\gserr_4$: $\mA \mC\ste_2, \mA^2\ste_2, \mA\ste_3,\ste_4$
  \item Terms appearing in $\deltaset_4$: $\mA \mC\ste_2, \mA^2\ste_2, \mA\ste_3,\ste_4, \mC\mA\ste_2, \mC\ste_3, \mC^2\ste_2, \ste_2\cdot\ste_2$

  \item Terms appearing in $\gserr_5$: \\ $\mA^2 \mC\ste_2, \mA^3\ste_2, \mA^2\ste_3,\mA\ste_4, \mA \mC\mA\ste_2, \mA \mC\ste_3, \mA \mC^2\ste_2, \mA (\ste_2\cdot\ste_2), \ste_5$

  \item Terms appearing in $\deltaset_5$: \\ $\mA^2 \mC\ste_2, \mA^3\ste_2, \mA^2\ste_3,\mA\ste_4, \mA \mC\mA\ste_2, \mA \mC\ste_3, \mA \mC^2\ste_2, \mA (\ste_2\cdot\ste_2), \ste_5, \\
  \mC \mA \mC\ste_2, \mC \mA^2\ste_2, \mC \mA\ste_3,\mC \ste_4, \\
  \mC^2 \mA \ste_2, \mC^2 \ste_3, \mC^3 \ste_2,
  \mC(\ste_2\cdot\ste_2), \ste_2\cdot\ste_3, \ste_2\cdot(\mA\ste_2)$

  \item Terms appearing in $\gserr_6$: \\ $\mA^3 \mC\ste_2, \mA^4\ste_2, \mA^3\ste_3,\mA^2\ste_4, \mA \mC\mA^2\ste_2, \mA^2 \mC\ste_3, \mA^2 \mC^2\ste_2, \mA^2 (\ste_2\cdot\ste_2), \mA \ste_5, \\
  \mA \mC \mA \mC\ste_2, \mA \mC \mA^2\ste_2, \mA \mC \mA\ste_3,\mA \mC \ste_4, \\
  \mA \mC^2 \mA \ste_2, \mA \mC^2 \ste_3, \mA \mC^3 \ste_2,
  \mA \mC(\ste_2\cdot\ste_2), \mA (\ste_2\cdot\ste_3), \mA (\ste_2\cdot(\mA\ste_2)), \ste_6$

  \item Terms appearing in $\deltaset_6$: \\ $\mA^3 \mC\ste_2, \mA^4\ste_2, \mA^3\ste_3,\mA^2\ste_4, \mA \mC\mA^2\ste_2, \mA^2 \mC\ste_3, \mA^2 \mC^2\ste_2, \mA^2 (\ste_2\cdot\ste_2), \mA \ste_5, \\
  \mA \mC \mA \mC\ste_2, \mA \mC \mA^2\ste_2, \mA \mC \mA\ste_3,\mA \mC \ste_4, \\
  \mA \mC^2 \mA \ste_2, \mA \mC^2 \ste_3, \mA \mC^3 \ste_2,
  \mA \mC(\ste_2\cdot\ste_2), \mA (\ste_2\cdot\ste_3), \mA (\ste_2\cdot(\mA\ste_2)), \ste_6, \\
  \mC \mA^2 \mC\ste_2, \mC \mA^3\ste_2, \mC \mA^2\ste_3, \mC \mA\ste_4, \mC \mA C\mA\ste_2, \mC \mA C\ste_3, \mC \mA C^2\ste_2, \mC \mA (\ste_2\cdot\ste_2), \mC \ste_5, \\
\mC^2 \mA C\ste_2, \mC^2 \mA^2\ste_2, \mC^2 \mA\ste_3, \mC^2 \ste_4, \mC^3\mA\ste_2, \mC^3\ste_3, \mC^4\ste_2, \\
\mC^2 \ste_2^2, \mC (\ste_2\cdot\ste_3), \mC(\ste_2\cdot(\mA\ste_2)),
\ste_2\cdot(\mA\mC\ste_2), \ste_2\cdot(\mA^2\ste_2), \ste_2\cdot(\mA\ste_3),
\ste_2\cdot\ste_4, \\
(\mA\ste_2)^2, \ste_3^2$
\end{itemize}
 
The number of order conditions grows rapidly with $k$.  The number of conditions
that must be considered can be reduced dramatically if the method is assumed to have
higher stage order.  So far we have assumed only that $\ste_1=0$, i.e. stage order one.
Assumption of stage order $q$ means simply that $\ste_k=0$ for $1\le k\le q$.
For example, assuming stage order three, we now give the order conditions for order seven.
\begin{itemize}
  \item Terms appearing in $\gserr_7$ that do not involve $\ste_2$ or $\ste_3$:
\\ $\mA^3\ste_4, \mA^2 \ste_5, \mA^2 \mC \ste_4, \mA \ste_6, \mA\mC\mA\ste_4, \mA\mC \ste_5, \mA\mC^2 \ste_4, \ste_7$

  \item Terms appearing in $\deltaset_7$ that do not involve $\ste_2$ or $\ste_3$:
\\ $\mA^3\ste_4, \mA^2 \ste_5, \mA^2 \mC \ste_4, \mA \ste_6, \mA\mC\mA\ste_4, \mA\mC \ste_5, \mA\mC^2 \ste_4, \ste_7, \\
\mC\mA^2\ste_4,\mC\mA\ste_5,\mC\mA\mC\ste_4,\mC\ste_6, \mC^2\mA\ste_4,\mC^2\ste_5,\mC^3\ste_4 $

\end{itemize}

\subsection{Order conditions}
Here we enumerate the order conditions themselves.

\begin{itemize}
  \item Order $p=1$: $\bb\transpose \bfe = 1$
  \item Order $p=2$: 
    \begin{itemize}
      \item $\bb\transpose \bc = \frac{1}{2}$
    \end{itemize}
  \item Order $p=3$:
    \begin{itemize}
      \item $\bb\transpose \bc^2 = \frac{1}{3}$
      \item $\bb\transpose \ste_2 = 0$
    \end{itemize}
  \item Order $p=4$:
    \begin{itemize}
      \item $\bb\transpose \bc^3 = \frac{1}{4}$
      \item $\bb\transpose \mC \ste_2 = 0$
      \item $\bb\transpose \mA \ste_2 = 0$
      \item $\bb\transpose \ste_3 = 0$
    \end{itemize}
  \item Order $p=5$:
    \begin{itemize}
      \item $\bb\transpose \bc^4 = \frac{1}{5}$
      \item $\bb\transpose \mA \mC \ste_2 = 0$
      \item $\bb\transpose \mA^2 \ste_2 = 0$
      \item $\bb\transpose \mA \ste_3 = 0$
      \item $\bb\transpose \ste_4 = 0$
      \item $\bb\transpose \mC \mA \ste_2 = 0$
      \item $\bb\transpose \mC \ste_3 = 0$
      \item $\bb\transpose \mC^2 \ste_2 = 0$
      \item $\bb\transpose \ste_2^2 = 0$
    \end{itemize}
  \item Order $p=6$:
    \begin{itemize}
      \item $\bb\transpose \bc^5 = \frac{1}{6}$
      \item $\bb\transpose \mA^2 \mC \ste_2 = 0$
      \item $\bb\transpose \mA^3 \ste_2 = 0$
      \item $\bb\transpose \mA^2 \ste_3 = 0$
      \item $\bb\transpose \mA \ste_4 = 0$
      \item $\bb\transpose \mA \mC \mA \ste_2 = 0$
      \item $\bb\transpose \mA \mC \ste_3 = 0$
      \item $\bb\transpose \mA \mC^2 \ste_2 = 0$
      \item $\bb\transpose \mA \ste_2^2 = 0$
      \item $\bb\transpose \ste_5 = 0$
      \item $\bb\transpose \mC \mA \mC \ste_2 = 0$
      \item $\bb\transpose \mC \mA^2 \ste_2 = 0$
      \item $\bb\transpose \mC \mA \ste_3 = 0$
      \item $\bb\transpose \mC \ste_4 = 0$
      \item $\bb\transpose \mC^2 \mA \ste_2 = 0$
      \item $\bb\transpose \mC^2 \ste_3 = 0$
      \item $\bb\transpose \mC^3 \ste_2 = 0$
      \item $\bb\transpose \mC \ste_2^2 = 0$
      \item $\bb\transpose (\ste_2\cdot\ste_3) = 0$
      \item $\bb\transpose (\ste_2 \cdot (\mA \ste_2)) = 0$
    \end{itemize}
\end{itemize}

The list of conditions that must be considered is smaller for methods with higher stage
order.  For instance, if we assume stage order two then
the conditions for fifth order are just
\begin{itemize}
  \item $\bb\transpose \bfe = 1$
  \item $\bb\transpose \bc = \frac{1}{2}$
  \item $\bb\transpose \bc^2 = \frac{1}{3}$
  \item $\bb\transpose \bc^3 = \frac{1}{4}$
  \item $\bb\transpose \ste_3 = 0$
  \item $\bb\transpose \bc^4 = \frac{1}{5}$
  \item $\bb\transpose \mA \ste_3 = 0$
  \item $\bb\transpose \ste_4 = 0$
  \item $\bb\transpose \mC \ste_3 = 0$
\end{itemize}
For stage order two methods, the additional conditions for 6th order are just
\begin{itemize}
  \item $\bb\transpose \bc^5 = \frac{1}{6}$
  \item $\bb\transpose \mA^2 \ste_3 = 0$
  \item $\bb\transpose \mA \ste_4 = 0$
  \item $\bb\transpose \mA \mC \ste_3 = 0$
  \item $\bb\transpose \ste_5 = 0$
  \item $\bb\transpose \mC \mA \ste_3 = 0$
  \item $\bb\transpose \mC \ste_4 = 0$
  \item $\bb\transpose \mC^2 \ste_3 = 0$
\end{itemize}
For stage order three methods, 
the additional conditions for 7th order are just
\begin{itemize}
  \item $\bb\transpose \bc^6 = \frac{1}{7}$
  \item $\bb\transpose \mA^2 \ste_4 = 0$
  \item $\bb\transpose \mA \ste_5 = 0$
  \item $\bb\transpose \mA\mC\ste_4 = 0$
  \item $\bb\transpose \ste_6 = 0$
  \item $\bb\transpose \mC\mA\ste_4 = 0$
  \item $\bb\transpose \mC\ste_5 = 0$
  \item $\bb\transpose \mC^2 \tau_4 = 0$
\end{itemize}
and the additional conditions for 8th order are just
\begin{itemize}
  \item $\bb\transpose \bc^7 = \frac{1}{8}$
  \item $\bb\transpose \mA^3\ste_4 = 0$
  \item $\bb\transpose \mA^2\ste_5 = 0$
  \item $\bb\transpose \mA^2\mC\ste_4 = 0$
  \item $\bb\transpose \mA\ste_6 = 0$
  \item $\bb\transpose \mA\mC\mA\ste_4 = 0$
  \item $\bb\transpose \mA\mC\ste_5 = 0$
  \item $\bb\transpose \mA\mC^2\ste_4 = 0$
  \item $\bb\transpose \ste_7 = 0$
  \item $\bb\transpose \mC\mA^2\ste_4 = 0$
  \item $\bb\transpose \mC\mA\ste_5 = 0$
  \item $\bb\transpose \mC\mA\mC\ste_4 = 0$
  \item $\bb\transpose \mC\ste_6 = 0$
  \item $\bb\transpose \mC^2\mA\ste_4 = 0$
  \item $\bb\transpose \mC^2\ste_5 = 0$
  \item $\bb\transpose \mC^3\ste_4 = 0$
\end{itemize}




\section{Order conditions for dense output}
Let us consider now continuous Runge--Kutta methods, which are equipped
with an additional set of weights $b_j(\theta)$ such that
\begin{align} \label{dense-output}
    u^{n+\theta} & = u^n + \Dt \sum_{j=1}^s b_j(\theta) f(y_j) \approx u(t_n + \theta \Dt).
\end{align}
Here we are generally interested in values $0\le \theta \le 1$, i.e.\ the
$\theta$-dependent weights are used to interpolate and provide output
at an arbitrary time within a given step.  Formula \eqref{dense-output}
is known as a {\em dense output} formula.  It avoids the need to ensure
that the integration hits desired output times exactly.  Most importantly,
it ensures that step sizes are determined only by accuracy and stability
considerations, even if very frequent output is required.

How accurate should the approximation \eqref{dense-output} be?
At first glance, it seems like the dense output formula ought
to have the same accuracy as that given by the regular weights (i.e., $p$).
However, since the dense output will not be used to propagate the solution,
and hence contributes only to local (not global) error, it is sufficient
to require order $p-1$.

What are the order conditions on the weights $b_j(\theta)$?  Here again
we will follow Albrecht's approach.  The standard references use Butcher's
approach and find that all of the order conditions become $\theta$-dependent.
But with Albrecht's approach we will see that only the bushy-tree conditions
depend on $\theta$.

To derive the order conditions, we follow the analysis of Section~\ref{sec:rkoc},
but in place of $u^{n+1}$ we consider the approximation $u^{n+\theta}$ above.
We find that 
\begin{align*}
\tu(t_{n+\theta}) & = \tu^{n} + \Dt \bb(\theta)\transpose\btf^n + \Dt \lte^n(\theta)
\end{align*}
where
\begin{align*}
\lte^n(\theta) & = \sum_{k=1}^\infty \lte_k \Dt^{k-1} \tu^{(k)}(t_{n})
\end{align*}
where
\begin{align*}
\lte_k & = \frac{\theta^k}{k!} - \frac{1}{(k-1)!} \bb\transpose \bc^{k-1}
\end{align*}
Notice that everything but the $\lte_k$ is unchanged from our earlier
analysis.  Thus the order conditions for order $p$ are
\begin{align*}    
\bb(\theta)\transpose \bc^{k-1} & = \frac{\theta^k}{k}  &  \mbox{for }  & 0\le k\le p \\
\bb(\theta)\transpose\rhsserr^n_k & = 0 &  \mbox{for } & 0\le k\le p-1.
\end{align*}
As predicted, only the bushy-tree conditions are modified.
Thus the vector $\bb(\theta)$ must lie, for every $\theta$, in the same
subspace determined by orthogonality to the vectors $\rhsserr_k$.


\bibliography{rk_order_conditions}

\end{document}
