\documentclass[12pt]{article}
\usepackage{graphicx}
\usepackage{amsmath,amssymb,amsthm}

\usepackage[T1]{fontenc}
\linespread{1.05}         % Palatino needs more leading (space between lines)
\usepackage[hmargin=1.0in]{geometry}

\bibliographystyle{apalike}

\begin{document}

\title{Order conditions for multistep Runge-Kutta methods}
\author{David I. Ketcheson} 
\maketitle


\newtheorem{thm}{Theorem}
\newtheorem{dfn}{Definition}
\newtheorem{lem}{Lemma}
\newtheorem{cor}{Corollary}
%\newtheorem{proposition}{Proposition}
%\newtheorem{result}{Result}
%\newtheorem{proof}{Proof}
%\newtheorem{conj}{Conjecture}
\newcommand{\qh}{\hat{q}}
\newcommand{\be}{\begin{equation}}
\newcommand{\ee}{\end{equation}}
\newcommand{\bq}{\mathbf{q}}
\newcommand{\bx}{\mathbf{x}}
\newcommand{\by}{\mathbf{y}}
\newcommand{\br}{\mathbf{r}}
\newcommand{\imh}{{i-\frac{1}{2}}}
\newcommand{\iph}{{i+\frac{1}{2}}}
\newcommand{\ipmh}{{i \pm \frac{1}{2}}}
\newcommand{\jph}{{j+\frac{1}{2}}}
\newcommand{\Aop}{{\cal A}}
\newcommand{\Bop}{{\cal B}}
\newcommand{\Wop}{{\cal W}}
\newcommand{\Oop}{{\cal O}}
\newcommand{\DQ}{\Delta Q}
\newcommand{\Dq}{\Delta q}
\newcommand{\Dx}{\Delta x}
\newcommand{\Dy}{\Delta y}
\newcommand{\Du}{\Delta u}
\newcommand{\bu}{\mathbf{u}}
\newcommand{\bv}{\mathbf{v}}
\newcommand{\bw}{\mathbf{w}}
\newcommand{\bU}{\mathbf{U}}
\newcommand{\bV}{\mathbf{V}}
\newcommand{\bF}{\mathbf{F}}
\newcommand{\Lop}{{\cal L}}
\newcommand{\Sop}{{\cal S}}
\newcommand{\Fop}{{\cal F}}
\newcommand{\Dofr}{{\cal D}(r)}
\newcommand{\Dt}{\Delta t}
\newcommand{\bbA}{\mathbf{A}}
\newcommand{\bbZ}{\mathbf{Z}}
\newcommand{\bbK}{\mathbf{K}}
\newcommand{\bbI}{\mathbf{I}}
\newcommand{\bbb}{\bar{\mathbf{b}}}
\newcommand{\bbe}{\mathbf{e}}
\newcommand{\bbone}{\mathbf{1}}
\newcommand{\lnorm}{\left\|}
\newcommand{\rnorm}{\right\|}

\newcommand{\dx}{\Delta x}
\newcommand{\dt}{\Delta t}
\newcommand{\aij}{\alpha_{i,j}}
\newcommand{\bij}{\beta_{i,j}}
\newcommand{\lt}{\tilde{L}}

\newcommand{\hf}{\frac{1}{2}}
\def\half{\frac{1}{2}}
\newcommand{\fracStrut}{\rule[-1.0ex]{0pt}{3.1ex}}
\newcommand{\hfs}{\ensuremath{\frac{1}{2}}\fracStrut}
\newcommand{\scinot}[2]{\ensuremath{#1\times10^{#2}}}
\newcommand{\dee}{\mathrm{d}}
\newcommand{\dye}{\partial}
\newcommand{\diff}[2]{\frac{\dee #1}{\dee #2}}
\newcommand{\pdiff}[2]{\frac{\dye #1}{\dye #2}}
\newcommand{\Real}{\mathbb{R}}
\newcommand{\Complex}{\mathbb{C}}
% matrices
\newcommand{\m}[1]{\mathbf{#1}}
\newcommand{\mA}{\m{A}}
\newcommand{\mAb}{\bar{\m{A}}}
\newcommand{\vah}{\hat{\v{a}}}
\newcommand{\mAh}{\hat{\m{A}}}
\newcommand{\mD}{\m{D}}
\newcommand{\mS}{\m{S}}
\newcommand{\mT}{\m{T}}
\newcommand{\mR}{\m{R}}
\newcommand{\mP}{\m{P}}
\newcommand{\mM}{\m{M}}
\newcommand{\mQ}{\m{Q}}
\newcommand{\mI}{\m{I}}
\newcommand{\mK}{\m{K}}
\newcommand{\mL}{\m{L}}
\newcommand{\mzero}{\m{0}}
% these use the upgreek package to get non-italic greek, which doesn't
% seem to work with \mathbf so these have to be setup manually to
% match \m
\newcommand{\matalpha}{\boldsymbol{\upalpha}}
\newcommand{\matbeta}{\boldsymbol{\upbeta}}
\newcommand{\mattheta}{\boldsymbol{\uptheta}}
\newcommand{\mateta}{\boldsymbol{\upeta}}
\newcommand{\matmu}{\boldsymbol{\upmu}}
\newcommand{\matlambda}{\boldsymbol{\uplambda}}
\newcommand{\matgamma}{\boldsymbol{\upgamma}}
\newcommand{\matdelta}{\boldsymbol{\updelta}}
% vectors
\renewcommand{\v}[1]{\boldsymbol{#1}}
\newcommand{\transpose}{^\mathrm{T}}
\newcommand{\thT}{\mattheta\transpose}
\newcommand{\bT}{\v{b}\transpose}
\newcommand{\bh}{\hat{b}}
\newcommand{\vbh}{\hat{\v{b}}}
\newcommand{\bhT}{\hat{\v{b}}\transpose}
\newcommand{\vb}{\v{b}}
\newcommand{\vc}{\v{c}}
\newcommand{\ve}{\v{e}}
\newcommand{\vu}{\v{u}}
\newcommand{\vl}{\boldsymbol{l}}
\newcommand{\vv}{\v{v}}
\newcommand{\vy}{\v{y}}
\newcommand{\vd}{\v{d}}
\newcommand{\va}{\v{a}}
\newcommand{\vf}{\v{f}}
\newcommand{\Matlab}{{\sc Matlab}\xspace}
\newcommand{\BARON}{{\sc Baron}\xspace}
\newcommand{\code}[1]{\textsf{#1}}
% the SSP coefficient
\newcommand{\sspcoef}{\mathcal{C}}
\newcommand{\clin}{\sspcoef_{\textup{lin}}}
\newcommand{\ceff}{\sspcoef_{\textup{eff}}}
\newcommand{\DtFE}{\Dt_{\textup{FE}}}

\newcommand{\Inc}{\textrm{Inc}}

\newcommand{\ty}{\tilde{y}}
\newcommand{\tu}{\tilde{u}}
\newcommand{\pt}{\tilde{p}}
\newcommand{\lte}{\tau}
\newcommand{\ste}{\boldsymbol{\tau}}
\newcommand{\bgamma}{\boldsymbol{\gamma}}
\newcommand{\btheta}{\boldsymbol{\theta}}
\newcommand{\bty}{\mathbf{\ty}}
\newcommand{\btu}{\mathbf{\tilde{u}}}

\newcommand{\gerr}{\delta u}
\newcommand{\gserr}{\delta\boldsymbol{y}}
\newcommand{\gsteperr}{\bar{\boldsymbol{\epsilon}}}
\newcommand{\rhserr}{\delta}
\newcommand{\rhsserr}{\delta\boldsymbol{f}}
\newcommand{\deltaset}{\bar{\boldsymbol{\delta}}}

\newcommand{\bff}{\mathbf{f}}
\newcommand{\bFf}{\mathbf{F}}
\newcommand{\bfb}{\mathbf{f}_u}
\newcommand{\bfbt}{\tilde{\mathbf{f}}_u}
\newcommand{\btf}{\mathbf{\tilde{f}}}
\newcommand{\bone}{\mathbf{1}}
\newcommand{\bb}{\mathbf{b}}
\newcommand{\bbh}{\hat{\mathbf{b}}}
\newcommand{\bc}{\mathbf{c}}
\newcommand{\bt}{\mathbf{t}}
\newcommand{\bg}{\mathbf{g}}
\newcommand{\bfe}{\mathbf{e}}
\newcommand{\mC}{\m{C}}

\renewcommand{\v}[1]{\mathbf{#1}}


\section{Review of Order Conditions for Runge--Kutta Methods}
  In this section we review the derivation of order conditions for Runge-Kutta
methods via the approach due to Albrecht \cite{albrecht1996}.
The basic notation is defined in section \ref{notation}.  A formula 
for the global error in terms of the method coefficients is derived
in section \ref{globerr}.  This leads to a recursion for determining the order 
conditions, as explained in section \ref{recursion}.

\subsection{Notation\label{notation}}
Write the $n$th step of a Runge--Kutta method as
\begin{subequations} \label{eq:rk}
\begin{align} 
\by^n & = u^{n}\ve + \Dt \mA \bff^{n} \\
u^{n+1} & = u^{n} + \Dt \bb\transpose \bff^{n}
\end{align}
\end{subequations}
where 
\begin{align*}
\by^n & = [y_1^n,\dots,y_s^n], \\
\bff^n & = [F(y_1^n),\dots,F(y_s^n)]
\end{align*}
are the vector of stage values and stage derivatives, respectively.
Let $u(t)$ denote the exact solution at time $t$ and define
the stage time vector $t^{n+c}_i = t^n+c_i\Dt$ and
the vectors of exact stage solution values and
exact stage derivatives:
\begin{align*}
    \by(\bt^{n+c}) & = [u(t^n+c_1\Dt),\dots,u(t^n+c_s\Dt)], \\
    \btf(\bt^{n+c}) & = [F(u(t^n+c_1\Dt)),\dots,F(u(t^n+c_s\Dt))].
\end{align*}
Next define the {\em truncation error} $\lte^n$ and
{\em stage truncation errors} $\ste^n$ by
\begin{subequations} \label{eq:rk_lte}
\begin{align} 
    \by(\bt^{n+c}) & = u(t^{n})\ve + \Dt \mA \btf(\bt^{n+c}) + \Dt \ste^n \\
    u(t_{n+1}) & = u(t^{n}) + \Dt \bb\transpose\btf(\bt^{n+c}) + \Dt \lte^n.
\end{align}
\end{subequations}

\subsection{Relation between the global and local errors\label{globerr}}
To find formulas for the truncation errors, we make use of 
the Taylor expansions
\begin{subequations} \label{eq:taylor}
\begin{align} 
u(t^{n}+c_i\Dt) & = \sum_{k=0}^\infty \frac{1}{k!} \Dt^k c_i^k u^{(k)}(t^{n}) \\
F(u(t^{n}+c_i\Dt)) & = \sum_{k=1}^\infty \frac{1}{(k-1)!} \Dt^{k-1} c_i^{k-1} u^{(k)}(t^{n})
\end{align}
\end{subequations}

Substitution of \eqref{eq:taylor} into \eqref{eq:rk_lte} gives
\begin{subequations} \label{eq:ste}
\begin{align}
\ste^n & = \sum_{k=1}^\infty \left( \frac{1}{k!} \bc^k - \frac{1}{(k-1)!} \mA \bc^{k-1}\right) \Dt^{k-1} u^{(k)}(t^{n}) = \sum_{k=1}^\infty \ste_k \Dt^{k-1} u^{(k)}(t^{n}) \\
\lte^n & = \sum_{k=1}^\infty \left( \frac{1}{k!} - \frac{1}{(k-1)!} \bb\transpose \bc^{k-1}\right) \Dt^{k-1} u^{(k)}(t^{n}) = \sum_{k=1}^\infty \lte_k \Dt^{k-1} u^{(k)}(t^{n})
\end{align}
\end{subequations} 
where 
\begin{subequations} 
\begin{align} \label{eq:ste_coeffs}
\ste_k & = \frac{1}{k!} \bc^k - \frac{1}{(k-1)!} \mA \bc^{k-1} \\
\lte_k & = \frac{1}{k!} - \frac{1}{(k-1)!} \bb\transpose \bc^{k-1}
\end{align}
\end{subequations} 

Subtracting \eqref{eq:rk_lte} from \eqref{eq:rk} gives
\begin{subequations} \label{eq:gserr_series}
\begin{align}
\gserr^n & = \gerr^{n} \ve + \Dt \mA \rhsserr^n - \Dt \ste^n \\
\gerr^{n+1} & = \gerr^{n} + \Dt \bb\transpose \rhsserr^n - \Dt \lte^n,
\end{align}
\end{subequations}
where $\gerr^{n+1}=u^{n+1}-u(t^{n+1})$ is the global error,
$\gserr^n = \by^n-\by(\bt^{n+c})$, is the global stage error, and
$\rhsserr^n = \bff^n-\bff(t^{n+c})$ is the stage derivative error.

Next assume expansions for the stage derivative errors $\rhsserr^n$ and
stage errors $\gserr^n$ as a power series in $\Dt$:
\begin{subequations} \label{eq:err_series}
\begin{align} \label{eq:rhserr_series}
\rhsserr^n & = \sum_{k=0}^{\infty} \rhsserr_k^n \Dt^{k}, \\
\gserr^n & = \sum_{k=0}^{\infty} \gserr_k^n \Dt^{k}.
\label{eq:gserr_power_series}
\end{align}
\end{subequations}
Then substituting the expansions \eqref{eq:rhserr_series} and \eqref{eq:ste}
into the global error formula \eqref{eq:gserr_series} yields
\begin{subequations} \label{eq:gserr_rec_both}
\begin{align} \label{eq:gserr_rec}
\gserr^n & = \gerr^n \ve + \sum_{k=0}^{p-1} \mA \rhsserr^n_k \Dt^{k+1} -\sum_{k=1}^{p} \ste_k u^{(k)}(t^{n}) \Dt^k + \Oop(\Dt^{p+1}) \\
\gerr^{n+1} & = \gerr^n + \sum_{k=0}^{p-1} \bb\transpose\rhsserr^n_{k-1} \Dt^{k+1} -\sum_{k=1}^{p} \lte_k u^{(k)}(t^{n}) \Dt^k + \Oop(\Dt^{p+1}) 
\end{align}
\end{subequations}
Assuming stable propagation of errors, we have global accuracy of order $p$
if the following conditions hold:
\begin{align*}    
\lte_k & = 0  & \mbox{for }  & 1\le k\le p \\
\bb\transpose\rhsserr^n_k & = 0 &  \mbox{for } & 0\le k\le p-1.
\end{align*}
The first set of conditions simply say that our quadrature rule would be of order $p$
if the stage values were exact.
The second set of conditions ensure that the errors due to the stage approximations
cancel out up to order $p$.

It remains to determine the vectors $\rhsserr^n_k$. In fact, 
we can relate these recursively to the global stage error vectors $\gserr_k$.  First define
\begin{align*}
\bFf(\by,\bt) & = [F(y_1(t^n+c_1 \Delta t)),\dots,F(y_s(t^n + c_s \Delta t))]\transpose.
\end{align*}
Then we have the Taylor series
\begin{align*}
    \bff^n =  \bFf(\by^n,\bt^{n+c}) & = \bff(\bt^{n+c}) + \sum_{j=1}^\infty \frac{1}{j!} (\by^n-\by(\bt^{n+c}))^j \cdot
    \bFf^{(j)}(\by(\bt^{n+c}),\bt^{n+c}) \\
    & = \bff(\bt^{n+c}) + \sum_{j=1}^\infty \frac{1}{j!} (\gserr^n)^j \cdot \bg_j(\bt^{n+c}),
\end{align*}
where 
\begin{align*}
\bFf^{(j)}(\by,\bt) = [F^{(j)}(y_1(t^n + c_1\Delta t)),\dots,F^{(j)}(y_s(t^n + c_s \Delta t))]\transpose, \\
\bg_j(\bt) = [F^{(j)}(y(t^n + c_1 \Delta t)),\dots,F^{(j)}(y(t^n + c_s \Delta t))]\transpose,
\end{align*}
and the dot product denotes componentwise multiplication. Thus
\begin{align}
  \rhsserr^n = \bff^n - \bff(\bt^{n+c}) = \sum_{j=1}^\infty \frac{1}{j!} (\gserr^n)^j \cdot
          \bg_j(\bt^{n+c}).
\end{align}
Since
\begin{align}
    \bg_j(\bt^{n+c}) & = \sum_{l=0}^\infty \frac{\Dt^l}{l!} \mC^l \bg_j^{(l)}(t^{n}),
\end{align}
where $\mC=\textup{diag}(\bc)$, we finally obtain the desired expansion:
\begin{align} \label{eq:triplesum}
    \sum_{k=0}^\infty \rhsserr^n_k & = \sum_{k=0}^\infty \sum_{j=1}^\infty \frac{\Dt^k}{j!k!} \mC^k (\gserr^n)^j \cdot \bg_j^{(k)}(t^{n}).
\end{align}

\subsection{Generation of stage derivative error vectors\label{recursion}}
Combining \eqref{eq:gserr_power_series} with \eqref{eq:gserr_rec} and equating
coefficients of powers of $\Dt$ gives (for $k\ge 1$)
\begin{align} \label{eq:gserrk}
\gserr^n_k & = \mA \rhsserr^n_{k-1} - \ste_k \tu^{(k)}(t^n).
\end{align}

To determine the coefficients $\rhsserr_k$, we alternate recursively 
between \eqref{eq:triplesum} and \eqref{eq:gserrk}.
Typically, the abscissas $\vc$ are chosen as $\mA\ve$ so that $\ste_1=0$; 
we will assume this since it simplifies the conditions considerably.

The terms appearing in the $\rhsserr_k$ involve products of certain constants
with derivatives of $\tu$ and the Butcher coefficients.  In order for $\bb\transpose \delta_k$
to vanish for arbitrary $\tu$, it must be that $\bb\transpose \bv = 0$ for each vector $\bv$
appearing in $\rhsserr_k$.  Since this latter condition does not depend on $|\bv|$,
the constants and the derivatives of $\tu$ can be neglected in our analysis.
Hence we focus solely on the vectors appearing in $\rhsserr_k$ depending on the 
Butcher coefficients.  We use the symbol $\deltaset_k$ to denote the set of these vectors.
Then the order conditions for order $p$ can be summarized as follows:
\begin{subequations}
\begin{align}\label{eq:bushy}
\frac{1}{k!} & = \frac{1}{(k-1)!}\bb\transpose \bc^{k-1} 
            & \mbox{for } 1\le k\le p \\
\bb\transpose\bv & = 0 & \mbox{for all } \bv\in\deltaset_k, \ \ \ \ \mbox{for } 1\le k\le p-1.
\label{eq:nonbushy}
\end{align}
\end{subequations}
%where $\bc=\mA\ve$ and
%\begin{align} 
%\ste_k & = \frac{1}{k!} \bc^k - \frac{1}{(k-1)!} \mA \bc^{k-1}.
%\end{align}
The conditions \eqref{eq:bushy} are referred to as {\em bushy-tree order conditions}
because they are associated with the bushy trees in Butcher's approach \cite{butcher2003}.
It is convenient to refer to the order conditions \eqref{eq:nonbushy} as {\em non-bushy-tree
order conditions}; the remainder of the section focuses on the task of determining these
explicitly.
Also $\rhsserr_0=0$ for any consistent method. Then taking $k=1$
in \eqref{eq:gserrk} gives $\gserr_1=0$.  Plugging this into \eqref{eq:triplesum}
yields $\rhsserr_1=0$.

Taking $k=2$ in \eqref{eq:gserrk}, we see that the factor $\ste_2$ appears in $\gserr_2$.
Plugging this into \eqref{eq:triplesum}, we see that $\ste_2$ appears in $\deltaset_2$.
Using this (with $k=3$) in \eqref{eq:gserrk}, we have that $\mA\ste_2$ and $\ste_3$
appear in $\gserr_3$.  Substituting this into \eqref{eq:triplesum} reveals that terms
proportional to $\mC\ste_2, \mA\ste_2, \ste_3$ appear in $\deltaset_3$.

Proceeding in this manner, the order conditions for any order of accuracy can be obtained.





\section{Enumeration of conditions}
  In this section we write out explicitly (for reference) the result of applying
the recursion derived in the previous section.

\subsection{Terms appearing in the error vectors}
Here we enumerate the terms generated by the recursion outlined above.
As before, we assume $\bc=\mA\bfe$ so that $\ste_1=0$.  
\begin{itemize}
  \item Terms appearing in $\deltaset_1$: $\emptyset$
  \item Terms appearing in $\gserr_2$: $\ste_2$
  \item Terms appearing in $\deltaset_2$: $\ste_2$
  \item Terms appearing in $\gserr_3$: $\mA\ste_2,\ste_3$
  \item Terms appearing in $\deltaset_3$: $\mC\ste_2, \mA\ste_2, \ste_3$
  \item Terms appearing in $\gserr_4$: $\mA \mC\ste_2, \mA^2\ste_2, \mA\ste_3,\ste_4$
  \item Terms appearing in $\deltaset_4$: $\mA \mC\ste_2, \mA^2\ste_2, \mA\ste_3,\ste_4, \mC\mA\ste_2, \mC\ste_3, \mC^2\ste_2, \ste_2\cdot\ste_2$

  \item Terms appearing in $\gserr_5$: \\ $\mA^2 \mC\ste_2, \mA^3\ste_2, \mA^2\ste_3,\mA\ste_4, \mA \mC\mA\ste_2, \mA \mC\ste_3, \mA \mC^2\ste_2, \mA (\ste_2\cdot\ste_2), \ste_5$

  \item Terms appearing in $\deltaset_5$: \\ $\mA^2 \mC\ste_2, \mA^3\ste_2, \mA^2\ste_3,\mA\ste_4, \mA \mC\mA\ste_2, \mA \mC\ste_3, \mA \mC^2\ste_2, \mA (\ste_2\cdot\ste_2), \ste_5, \\
  \mC \mA \mC\ste_2, \mC \mA^2\ste_2, \mC \mA\ste_3,\mC \ste_4, \\
  \mC^2 \mA \ste_2, \mC^2 \ste_3, \mC^3 \ste_2,
  \mC(\ste_2\cdot\ste_2), \ste_2\cdot\ste_3, \ste_2\cdot(\mA\ste_2)$

  \item Terms appearing in $\gserr_6$: \\ $\mA^3 \mC\ste_2, \mA^4\ste_2, \mA^3\ste_3,\mA^2\ste_4, \mA \mC\mA^2\ste_2, \mA^2 \mC\ste_3, \mA^2 \mC^2\ste_2, \mA^2 (\ste_2\cdot\ste_2), \mA \ste_5, \\
  \mA \mC \mA \mC\ste_2, \mA \mC \mA^2\ste_2, \mA \mC \mA\ste_3,\mA \mC \ste_4, \\
  \mA \mC^2 \mA \ste_2, \mA \mC^2 \ste_3, \mA \mC^3 \ste_2,
  \mA \mC(\ste_2\cdot\ste_2), \mA (\ste_2\cdot\ste_3), \mA (\ste_2\cdot(\mA\ste_2)), \ste_6$

  \item Terms appearing in $\deltaset_6$: \\ $\mA^3 \mC\ste_2, \mA^4\ste_2, \mA^3\ste_3,\mA^2\ste_4, \mA \mC\mA^2\ste_2, \mA^2 \mC\ste_3, \mA^2 \mC^2\ste_2, \mA^2 (\ste_2\cdot\ste_2), \mA \ste_5, \\
  \mA \mC \mA \mC\ste_2, \mA \mC \mA^2\ste_2, \mA \mC \mA\ste_3,\mA \mC \ste_4, \\
  \mA \mC^2 \mA \ste_2, \mA \mC^2 \ste_3, \mA \mC^3 \ste_2,
  \mA \mC(\ste_2\cdot\ste_2), \mA (\ste_2\cdot\ste_3), \mA (\ste_2\cdot(\mA\ste_2)), \ste_6, \\
  \mC \mA^2 \mC\ste_2, \mC \mA^3\ste_2, \mC \mA^2\ste_3, \mC \mA\ste_4, \mC \mA C\mA\ste_2, \mC \mA C\ste_3, \mC \mA C^2\ste_2, \mC \mA (\ste_2\cdot\ste_2), \mC \ste_5, \\
\mC^2 \mA C\ste_2, \mC^2 \mA^2\ste_2, \mC^2 \mA\ste_3, \mC^2 \ste_4, \mC^3\mA\ste_2, \mC^3\ste_3, \mC^4\ste_2, \\
\mC^2 \ste_2^2, \mC (\ste_2\cdot\ste_3), \mC(\ste_2\cdot(\mA\ste_2)),
\ste_2\cdot(\mA\mC\ste_2), \ste_2\cdot(\mA^2\ste_2), \ste_2\cdot(\mA\ste_3),
\ste_2\cdot\ste_4, \\
(\mA\ste_2)^2, \ste_3^2$
\end{itemize}
 
The number of order conditions grows rapidly with $k$.  The number of conditions
that must be considered can be reduced dramatically if the method is assumed to have
higher stage order.  So far we have assumed only that $\ste_1=0$, i.e. stage order one.
Assumption of stage order $q$ means simply that $\ste_k=0$ for $1\le k\le q$.
For example, assuming stage order three, we now give the order conditions for order seven.
\begin{itemize}
  \item Terms appearing in $\gserr_7$ that do not involve $\ste_2$ or $\ste_3$:
\\ $\mA^3\ste_4, \mA^2 \ste_5, \mA^2 \mC \ste_4, \mA \ste_6, \mA\mC\mA\ste_4, \mA\mC \ste_5, \mA\mC^2 \ste_4, \ste_7$

  \item Terms appearing in $\deltaset_7$ that do not involve $\ste_2$ or $\ste_3$:
\\ $\mA^3\ste_4, \mA^2 \ste_5, \mA^2 \mC \ste_4, \mA \ste_6, \mA\mC\mA\ste_4, \mA\mC \ste_5, \mA\mC^2 \ste_4, \ste_7, \\
\mC\mA^2\ste_4,\mC\mA\ste_5,\mC\mA\mC\ste_4,\mC\ste_6, \mC^2\mA\ste_4,\mC^2\ste_5,\mC^3\ste_4 $

\end{itemize}

\subsection{Order conditions}
Here we enumerate the order conditions themselves.

\begin{itemize}
  \item Order $p=1$: $\bb\transpose \bfe = 1$
  \item Order $p=2$: 
    \begin{itemize}
      \item $\bb\transpose \bc = \frac{1}{2}$
    \end{itemize}
  \item Order $p=3$:
    \begin{itemize}
      \item $\bb\transpose \bc^2 = \frac{1}{3}$
      \item $\bb\transpose \ste_2 = 0$
    \end{itemize}
  \item Order $p=4$:
    \begin{itemize}
      \item $\bb\transpose \bc^3 = \frac{1}{4}$
      \item $\bb\transpose \mC \ste_2 = 0$
      \item $\bb\transpose \mA \ste_2 = 0$
      \item $\bb\transpose \ste_3 = 0$
    \end{itemize}
  \item Order $p=5$:
    \begin{itemize}
      \item $\bb\transpose \bc^4 = \frac{1}{5}$
      \item $\bb\transpose \mA \mC \ste_2 = 0$
      \item $\bb\transpose \mA^2 \ste_2 = 0$
      \item $\bb\transpose \mA \ste_3 = 0$
      \item $\bb\transpose \ste_4 = 0$
      \item $\bb\transpose \mC \mA \ste_2 = 0$
      \item $\bb\transpose \mC \ste_3 = 0$
      \item $\bb\transpose \mC^2 \ste_2 = 0$
      \item $\bb\transpose \ste_2^2 = 0$
    \end{itemize}
  \item Order $p=6$:
    \begin{itemize}
      \item $\bb\transpose \bc^5 = \frac{1}{6}$
      \item $\bb\transpose \mA^2 \mC \ste_2 = 0$
      \item $\bb\transpose \mA^3 \ste_2 = 0$
      \item $\bb\transpose \mA^2 \ste_3 = 0$
      \item $\bb\transpose \mA \ste_4 = 0$
      \item $\bb\transpose \mA \mC \mA \ste_2 = 0$
      \item $\bb\transpose \mA \mC \ste_3 = 0$
      \item $\bb\transpose \mA \mC^2 \ste_2 = 0$
      \item $\bb\transpose \mA \ste_2^2 = 0$
      \item $\bb\transpose \ste_5 = 0$
      \item $\bb\transpose \mC \mA \mC \ste_2 = 0$
      \item $\bb\transpose \mC \mA^2 \ste_2 = 0$
      \item $\bb\transpose \mC \mA \ste_3 = 0$
      \item $\bb\transpose \mC \ste_4 = 0$
      \item $\bb\transpose \mC^2 \mA \ste_2 = 0$
      \item $\bb\transpose \mC^2 \ste_3 = 0$
      \item $\bb\transpose \mC^3 \ste_2 = 0$
      \item $\bb\transpose \mC \ste_2^2 = 0$
      \item $\bb\transpose (\ste_2\cdot\ste_3) = 0$
      \item $\bb\transpose (\ste_2 \cdot (\mA \ste_2)) = 0$
    \end{itemize}
\end{itemize}

The list of conditions that must be considered is smaller for methods with higher stage
order.  For instance, if we assume stage order two then
the conditions for fifth order are just
\begin{itemize}
  \item $\bb\transpose \bfe = 1$
  \item $\bb\transpose \bc = \frac{1}{2}$
  \item $\bb\transpose \bc^2 = \frac{1}{3}$
  \item $\bb\transpose \bc^3 = \frac{1}{4}$
  \item $\bb\transpose \ste_3 = 0$
  \item $\bb\transpose \bc^4 = \frac{1}{5}$
  \item $\bb\transpose \mA \ste_3 = 0$
  \item $\bb\transpose \ste_4 = 0$
  \item $\bb\transpose \mC \ste_3 = 0$
\end{itemize}
For stage order two methods, the additional conditions for 6th order are just
\begin{itemize}
  \item $\bb\transpose \bc^5 = \frac{1}{6}$
  \item $\bb\transpose \mA^2 \ste_3 = 0$
  \item $\bb\transpose \mA \ste_4 = 0$
  \item $\bb\transpose \mA \mC \ste_3 = 0$
  \item $\bb\transpose \ste_5 = 0$
  \item $\bb\transpose \mC \mA \ste_3 = 0$
  \item $\bb\transpose \mC \ste_4 = 0$
  \item $\bb\transpose \mC^2 \ste_3 = 0$
\end{itemize}
For stage order three methods, 
the additional conditions for 7th order are just
\begin{itemize}
  \item $\bb\transpose \bc^6 = \frac{1}{7}$
  \item $\bb\transpose \mA^2 \ste_4 = 0$
  \item $\bb\transpose \mA \ste_5 = 0$
  \item $\bb\transpose \mA\mC\ste_4 = 0$
  \item $\bb\transpose \ste_6 = 0$
  \item $\bb\transpose \mC\mA\ste_4 = 0$
  \item $\bb\transpose \mC\ste_5 = 0$
  \item $\bb\transpose \mC^2 \tau_4 = 0$
\end{itemize}
and the additional conditions for 8th order are just
\begin{itemize}
  \item $\bb\transpose \bc^7 = \frac{1}{8}$
  \item $\bb\transpose \mA^3\ste_4 = 0$
  \item $\bb\transpose \mA^2\ste_5 = 0$
  \item $\bb\transpose \mA^2\mC\ste_4 = 0$
  \item $\bb\transpose \mA\ste_6 = 0$
  \item $\bb\transpose \mA\mC\mA\ste_4 = 0$
  \item $\bb\transpose \mA\mC\ste_5 = 0$
  \item $\bb\transpose \mA\mC^2\ste_4 = 0$
  \item $\bb\transpose \ste_7 = 0$
  \item $\bb\transpose \mC\mA^2\ste_4 = 0$
  \item $\bb\transpose \mC\mA\ste_5 = 0$
  \item $\bb\transpose \mC\mA\mC\ste_4 = 0$
  \item $\bb\transpose \mC\ste_6 = 0$
  \item $\bb\transpose \mC^2\mA\ste_4 = 0$
  \item $\bb\transpose \mC^2\ste_5 = 0$
  \item $\bb\transpose \mC^3\ste_4 = 0$
\end{itemize}




\section{Multistep Runge--Kutta Methods of Type I}
  We consider multistep Runge--Kutta methods of the form
\begin{subequations} \label{eq:mrk-full}
\begin{align}
y_i^n & = \sum_{l=1}^{k} d_{il} u^{n-k+l} + \Dt\sum_{j=1}^s a_{ij} F(y_j^n) \\
u^{n+1} & = \sum_{l=1}^{k} \theta_l u^{n-k+l} + \Dt\sum_{j=1}^s b_j F(y_j^n),
\end{align}
\end{subequations}
where it is assumed that (for consistency)
$$\sum_{l=1}^k d_{il} = 1 \ \ \ \ \sum_{l=1}^k \theta_l = 1.$$
This form includes both Type I and Type II methods, but it must be remembered
that $s$ is not an accurate indication of the cost of the method for
Type II methods in this form.  More specifically,
it is possible to include terms such as $F(u^{n-1})$ by having
one of the stages equal to $u^{n-1}$ identically (like the Type II TSRK
methods from our paper), and that in this
case the cost of the method is generally less than $s$ function 
evaluations.

We write method \eqref{eq:mrk-full} as
\begin{subequations} \label{eq:mrk}
\begin{align} 
\by^n & = \mD \bu^n + h \mA \bff^n \\
u^{n+1} & = \btheta \bu^n + h \bb\transpose \bff^n
\end{align}
\end{subequations}
where $\bu$ is the vector of previous step values 
$$\bu = [u^{n-k+1}, u^{n-k+2},\dots, u^{n}].$$
Then the true solution satisfies
\begin{subequations} \label{eq:mrk_lte}
\begin{align} 
\bty^n & = \mD \btu^n + h \mA \btf^n + h \ste^n \\
\tu^{n+1} & = \btheta \btu^n + h \bb\transpose\btf^n + h \lte^n.
\end{align}
\end{subequations}

Using the Taylor expansions above, as well as
\begin{align*}
\tu(t_{n-1-r}) & = \tu(t_{n-1}) - r h \tu'(t_{n-1}) + \frac{1}{2} h^2 r^2 \tu''(t_{n-1}) + \cdots \\
& = \sum_{k=0}^\infty \frac{1}{k!} h^k (-r)^k \tu^{(k)}(t_{n-1}),
\end{align*}
substitution gives
\begin{subequations} \label{eq:mrk_ste}
\begin{align}
\ste & = \sum_{k=1}^\infty \ste_k h^{k-1} \tu^{(k)}(t_{n-1}) \\
\lte^n & = \sum_{k=1}^\infty \lte_k h^{k-1} \tu^{(k)}(t_{n-1})
\end{align}
\end{subequations} 
where
\begin{subequations} 
\begin{align} \label{eq:mrk_ste_coeffs}
\ste_k & = \frac{1}{k!} \left(\bc^k-\mD(-\vl)^k\right) - \frac{1}{(k-1)!} \mA \bc^{k-1} \\
\lte_k & = \frac{1}{k!}\left(1-\btheta\transpose(-\vl)^k\right) - \frac{1}{(k-1)!} \bb\transpose \bc^{k-1}
\label{eq:mrk_te_coeffs}
\end{align}
\end{subequations} 
and $\vl=[k-1,k-2,\dots,0].$  We assume consistency of the stages, which
means $\ste_1=0$.  We take this condition to define the abscissas, which means we
have stage order at least equal to one:
\begin{align} \label{eq:mrk_cdef} 
\bc & = \mA\ve -\mD\vl.
\end{align}

Subtracting \eqref{eq:mrk_lte} from \eqref{eq:mrk} gives
\begin{subequations} \label{eq:mrk_gserr_series}
\begin{align}
\gserr^n & = \mD \gsteperr^n + h \mA \rhsserr^n - h \ste^n \\
\gerr^{n+1} & = \btheta\transpose \gsteperr^n + h \bb\transpose \rhsserr^n - h \lte^n
\end{align}
\end{subequations}
where $\gsteperr^n = [\gerr^{n-1},\gerr^{n-2},\dots,\gerr^{n-k}]$.

Now we seek expressions for the global stage errors $\gserr$ and
the stage derivative errors $\rhsserr$ of the form \eqref{eq:err_series}.
%Since $\gsteperr=\Oop(h^p)$,
Substituting \eqref{eq:rhserr_series} and \eqref{eq:mrk_ste}
into \eqref{eq:mrk_gserr_series} yields %again \eqref{eq:gserr_rec},
\begin{subequations} \label{eq:gserr_rec_mrk_both}
\begin{align} \label{eq:gserr_rec_mrk}
\gserr^n & =\mD \gsteperr^n + \sum_{k=0}^{p-1} \mA \rhsserr^n_k \Dt^{k+1} -\sum_{k=1}^{p} \ste_k \tu^{(k)}(t_{n}) \Dt^k + \Oop(\Dt^{p+1}) \\
\gerr^{n+1} & = \btheta\transpose \gsteperr^n + \sum_{k=0}^{p-1} \bb\transpose\rhsserr^n_{k-1} \Dt^{k+1} -\sum_{k=1}^{p} \lte_k \tu^{(k)}(t_{n}) \Dt^k + \Oop(\Dt^{p+1}) 
\end{align}
\end{subequations}
with $\ste_k$ given by \eqref{eq:mrk_ste_coeffs}.
Assuming stable propagation of errors, we again have global accuracy of order $p$
if 
\begin{align*}    
\lte_k & = 0  & \mbox{for }  & 0\le k\le p \\
\bb\transpose\rhsserr^n_k & = 0 &  \mbox{for } & 0\le k\le p-1.
\end{align*}
Furthermore, we still have the expression \eqref{eq:triplesum} for $\rhsserr$.

\subsection{Generation of stage derivative error vectors\label{recursion}}
Combining \eqref{eq:gserr_series} with \eqref{eq:gserr_rec_mrk} and equating
coefficients of powers of $\Dt$ gives again \eqref{eq:gserrk}.
Hence we can again determine the vectors appearing in $\rhsserr_k$ recursively using
\eqref{eq:triplesum} and \eqref{eq:gserrk}.  The only difference is that the stage truncation
error vectors are now given by \eqref{eq:mrk_ste_coeffs}.  
Hence the non-bushy tree order conditions for these methods are the same as those for
Runge-Kutta methods, as enumerated in Section 2, except that the definitions of the stage truncation
errors $\ste_k, \lte_k,$ and of the abscissas $\bc$ are given by \eqref{eq:mrk_ste_coeffs},
\eqref{eq:mrk_te_coeffs}, and \eqref{eq:mrk_cdef}, respectively.
Meanwhile, the bushy tree order conditions for order $p$ are (from \eqref{eq:mrk_te_coeffs}):
\begin{align*}
\frac{1}{k!}\left(1-\btheta\transpose(-\vl)^k\right) & = \frac{1}{(k-1)!}\bb\transpose \bc^{k-1} 
            & \mbox{for } 1\le k\le p.%\\
%\bb\transpose\bv & = 0 & \mbox{for all } \bv\in\rhsserr_k, \ \ \ \ \mbox{for } 1\le k\le p,
\end{align*}
%where
%\begin{align}
%\bc & = \mA\ve -\mD\vl,
%\end{align}
%\begin{align} 
%\ste_k & = \frac{1}{k!} \left(\bc^k-\mD(-\vl)^k\right) - \frac{1}{(k-1)!} \mA \bc^{k-1},
%\end{align}
%and the $\rhsserr_k$ contain the following vectors:




\section{Spijker forms}
The Spijker form for the Type I methods of the last section is
\begin{align}
\bx & = \left[u^{n-k+1},u^{n-k+2},\dots,u^{n-1},u^n\right] \\
\by & = \left[u^{n-k+1},u^{n-k+2},\dots,u^{n-1},u^n,y_1,\dots,y_s,u^{n+1}\right] \\
\mS & = \begin{pmatrix}\mI\\\mD\\\btheta\transpose\end{pmatrix} \ \ \ \
\mT   = \begin{pmatrix}\mzero & \mzero & 0 \\ \mzero & \mA & \mzero \\ \mzero & \bb\transpose & 0 \end{pmatrix}.
\end{align}

We are also interested in Type II methods, which take the form
\begin{subequations} \label{eq:mrk-typeII}
\begin{align}
y_i^n & = \sum_{l=1}^{k} d_{il} u^{n-k+l} + \Dt\sum_{l=1}^{k-1} \hat{a}_{il} F(u^{n-k+l}) + \Dt\sum_{j=1}^s a_{ij} F(y_j^n) \\
u^{n+1} & = \sum_{l=1}^{k} \theta_l u^{n-k+l} + \Dt\sum_{l=1}^{k-1} \hat{b}_{l} F(u^{n-k+l}) + \Dt\sum_{j=1}^s b_j F(y_j^n),
\end{align}
\end{subequations}
where $y_1^n=u^n$.  These admit the Spijker form
\begin{align}
\bx & = \left[u^{n-k+1},u^{n-k+2},\dots,u^{n-1},u^n\right] \\
\by & = \left[u^{n-k+1},u^{n-k+2},\dots,u^{n-1},u^n,y_1=u^{n},y_2,\dots,y_s,u^{n+1}\right] \\
\mS & = \begin{pmatrix}\mI \ \mzero \\ \mD \\\btheta\transpose \end{pmatrix} \ \ \ \
\mT   = \begin{pmatrix}\mzero & \mzero & 0 \\ \mAh & \mA & \mzero \\ \bbh\transpose & \bb\transpose & 0 \end{pmatrix}.
\end{align}
Here the first row of $\mD$ is $(0,0,\dots,0,1)$ and the first row of
$\mA,\mAh$ is identically zero.
The order conditions presented above can also be applied to these Type II methods
by writing them as Type I methods (simply introduce additional stages that are
equal to the old steps).  {\bf David can write this last bit up if necessary.}

\section{Methods for linear problems}
In this section we consider only linear problems of the form $$u_t = L u,$$
so that the order conditions simplify considerably.  This case is of 
interest for several reasons.  First, linear problems are of interest in many 
cases (cite Bernardo Cockburn), and it is possible to include a time-dependent 
forcing term or boundary condition while maintaining the order of the method. 
Second, up to second order  the order conditions for  linear and nonlinear
problems are the same.  Thus, this approach will provide results for second 
order multi-step multi-stage methods, which gives us a picture of the size 
of the time-step for higher order nonlinear methods.  Finally, the time-step 
restriction for the linear high order methods serves as an upper bound for 
that of nonlinear high order methods. 

\subsection{Tall Tree Order conditions for Multistep Runge--Kutta methods}
For $k$-step Runge-Kutta methods of Type II,
applying the method to the linear scalar homogeneous test equation, we
find that it reduces to the iteration
\be
u^{n+1} = \left( \btheta\transpose + z \hat{\vb}\transpose + z \vb\transpose ( I-z\mA)^{-1} (\mD+z\hat{\mA}) \right) \vu
\ee
where
\be
\vu = \left(u^{n-k+1},u^{n-k+2},\dots,u^{n}\right)^T
\ee
and $\mAh,\vbh\transpose$ each have $k$ columns but the last column is zero.
Using the Taylor expansions 
\begin{subequations}
\begin{align}
\left( \mI - z\mA \right)^{-1}  & = \sum_{j=0}^\infty z^j \mA^j \\
u^{n-\vl} = e^{-\vl\Dt}u^n  & = u^n \sum_{j=0}^\infty \frac{1}{j!} z^j (-l)^j
\end{align}
\end{subequations}
we obtain the tall tree order conditions for linear problems.
These are given by equating coefficients in
\begin{align}
e^z + \Oop(z^{p+1}) & = 
    \left[ \btheta\transpose + z \bhT + z \bT
    \left(\sum_{j=0}^\infty z^j \mA^j \right) (z + \mD \hat{\mA})\right]
    \left(\sum_{j=0}^\infty \frac{1}{j!}z^j (-\vl)^j\right) \nonumber \\
 & = \sum_{j=0}^\infty z^j {\huge(} \frac{(-1)^j}{j!} \btheta\transpose \vl^j
    + \frac{(-1)^{j-1}}{(j-1)!}\bhT \vl^{j-1} \\
    & + \bT \sum_{i=0}^{j-1}\frac{(-1)^i}{i!}\mA^{j-i-1}\mD\vl^i
    + \frac{(-1)^{i-1}}{(i-1)!}\mA^{j-i}\hat{\mA}\vl^{i-1} \nonumber
    {\huge)}
\end{align}
where $\vl=(k-1,k-2,\dots,0)\transpose$.
{\em need to add something about terms with negative exponents
being understood to be zero here}

The resulting conditions are (given our assumption that $\mD\ve=\ve$ and 
$\btheta\transpose\ve=1$):
\begin{subequations}
\begin{align}
(\bT + \bhT)\ve - \btheta\transpose\vl & = 1 \\
\bT\vc - \bhT\vl
    + \frac{1}{2}\btheta\transpose\vl^2 & = \frac{1}{2} \\
\bT(\mA\vc + \frac{1}{2}\mD\vl^2 - \mAh\vl) +\frac{1}{2}\bhT\vl^2 
    - \frac{1}{6}\btheta\transpose\vl^3 & = \frac{1}{6}
\end{align}
\end{subequations}
where
$$ \vc = (\mA+\hat{\mA})\ve - \mD\vl.$$


For $k$-step Runge-Kutta methods of Type I,
the analysis is the same as that above, but with
$\hat{\vb}=\mAh=0$.
The first five order conditions in this case reduce to
\begin{subequations}
\label{2ndOCs}
\begin{align}
\vb^T \ve - \btheta\transpose \vl &= 1, \\
\vb^T \vc + \frac{1}{2} \btheta\transpose \vl^2 & = \frac{1}{2}, \\
\vb\transpose \left(\mA\vc + \frac{1}{2}\mD\vl^2\right) - \frac{1}{6} \btheta\transpose\vl^3 & = \frac{1}{6} \\
\vb\transpose \left(\mA^2\vc + \frac{1}{2}\mA\mD\vl^2 - \frac{1}{6}\mD\vl^3\right) + \frac{1}{24} \btheta\transpose\vl^4 & = \frac{1}{24} \\
\vb\transpose \left(\mA^3\vc + \frac{1}{2}\mA^2\mD\vl^2 - \frac{1}{6}\mA\mD\vl^3 + \frac{1}{24}\mD\vl^4\right) - \frac{1}{120} \btheta\transpose\vl^5 & = \frac{1}{120}
\end{align}
\end{subequations}
where $\vc=\mA\ve-\mD\vl$.


\subsection{Optimal SSP Multistep Runge--Kutta methods for linear problems}

{\bf These are David's old results from 2007-2008.}
SSP coefficients for optimal explicit methods of Type II are shown in Tables
\ref{tbl-2order}-\ref{tbl-3order}.
We have solved numerically the optimization problem of maximizing $r$ subject to
(\ref{2ndOCs}) and (\ref{spijkermon}) for various values of $s,k$ with $p=2$
(i.e., implicit methods of Type I).
It appears (after some \verb fmincon  searches) that the optimal 2nd order 
SSP implicit Runge-Kutta methods given in \cite{ferracina2008,ketcheson2008b} 
are actually optimal over all second order $k$-step, 2-stage methods.

\begin{table}
\center
\begin{tabular}{l|cccc} \hline
s \ k & 2     & 3     & 4     & 5 \\ \hline
2 & 0.707 & 0.809 & 0.860 &  \\
3 & 0.817 & 0.879 & 0.911* & \\
4 & 0.866 & 0.911 & 0.934 & \\
5 & 0.894 & 0.930 &  & \\  \hline
\end{tabular}
\caption{Effective SSP coefficients $\ceff$ of optimal
  explicit 2nd order $k$-step Runge-Kutta methods of type (explicit with
  a zero row) (for both linear and nonlinear problems).}
\label{tbl-2order}
\end{table}


\begin{table}
\center
\begin{tabular}{l|cccc} \hline
s \ k & 2     & 3     & 4     & 5 \\ \hline
2 & 0.366 & 0.556 & 0.622 & 0.622 \\
3 & 0.550 & 0.667*& 0.667 & \\
4 & 0.627 &       &       & \\
5 & 0.677 & 0.679 & 0.679 & \\ 
6 & 0.705 & 0.705 & 0.705      & \\ \hline
\end{tabular}
\caption{Effective SSP coefficients $\ceff$ of optimal
  explicit 3rd order $k$-step Runge-Kutta methods of type (explicit with
  a zero row) for linear problems.}
\label{tbl-3order}
\end{table}



\bibliography{rk_order_conditions}

\end{document}
